\chapter{Quantum Mechanics} \label{ch:qm}

\section{State Vectors and Dirac Notation}

In quantum mechanics everything knowable about the state of some system is described in a vector, known as the state vector. The vector is from a vector space defined over the field of complex numbers, so it is important to use the correct definition of the inner product (§\ref{ch:vectors-complex}) where we take the conjugate of one of the vectors, to ensure that the inner product of a vector with itself is a non-negative a real number.

The inner product in this context is written like this:

$$\langle a|b \rangle$$

If the vectors $\vec{a}$ and $\vec{b}$ are represented by column matrices $a$ and $b$ (to spare ourselves, for the moment, from things we can't imagine, let's pretend we're discussing a finite-dimensional vector space), the above is equivalent to conjugate-transpose of $a$, written as $a^{\dagger}$ ("a-dagger"), matrix-multiplied by $b$:

$$a^{\dagger} \, b$$

We can split this inner product notation into separate pieces, so we can write $\langle a|$ to mean the vector whose matrix representation in some basis is a single row containing the complex conjugates of the elements in the single column of the matrix representing $|a \rangle$ in the corresponding dual basis.

Or more simply, $\langle a|$ is the covector of $|a \rangle$. The convention is therefore to think of $\langle a|$ as a function that extracts the coordinate of a basis vector $|a \rangle$ from its argument, which will be some vector $|b \rangle$, as in the expression $\langle a|b \rangle$. And in concrete matrix terms we can picture $\langle a|$ as a 1-row matrix (a row vector) that is the dual of the 1-column matrix (column vector) $|a \rangle$, and their corresponding coordinates are mutually complex conjugates.

And with this in mind, it follows that we can write them the other way round from the inner product:

$$|b \rangle \langle a |$$

which must therefore define a matrix: the product of a column vector on the left and a row vector on the right. This is the \textit{outer} product. A matrix can act as a vector-valued function of vectors: apply it to a vector to transform that vector to another vector. So:

$$|b \rangle \langle a | c \rangle$$

See how the notation nicely suggests we bracket the $\langle a | c \rangle$ first as an inner product and thus a mere number. So we immediately know that the result will be the vector $|b \rangle$ scaled by a number, i.e. it will be co-linear with $|b \rangle$. We've measured $|c \rangle$ against $|a \rangle$ and used that to scale $|b \rangle$.

Given an orthonormal basis $|b_n \rangle$, we can picture it as a set of $n$ column vectors, and expressed in their own basis they would be the standard basis. The outer product:

$$|b_n \rangle \langle b_n |$$

will produce a matrix with a single $1$ in one place of the diagonal. So if we sum over all $n$, we get the identity matrix, a matrix that makes no difference to whatever vector it applies to. Spelling this out, if our vector space has just two basis vectors:

$$|0 \rangle = \begin{bmatrix} 1 \\ 0 \end{bmatrix}$$

and

$$|1 \rangle = \begin{bmatrix} 0 \\ 1 \end{bmatrix}$$

Then the outer product of $|0 \rangle$ with itself is just:

$$
|0 \rangle \langle 0| = 
\begin{bmatrix} 1 \\ 0 \end{bmatrix}
\begin{bmatrix} 1 & 0 \end{bmatrix} =
\begin{bmatrix} 1 & 0 \\ 0 & 0 \end{bmatrix}
$$

and likewise of $|1 \rangle$ with itself:

$$
|1 \rangle \langle 1| = 
\begin{bmatrix} 0 \\ 1 \end{bmatrix}
\begin{bmatrix} 0 & 1 \end{bmatrix} =
\begin{bmatrix} 0 & 0 \\ 0 & 1 \end{bmatrix}
$$

And as predicted, summing those matrices gives the identity matrix:

$$
\begin{bmatrix} 1 & 0 \\ 0 & 0 \end{bmatrix} +
\begin{bmatrix} 0 & 0 \\ 0 & 1 \end{bmatrix} =
\begin{bmatrix} 1 & 0 \\ 0 & 1 \end{bmatrix}
$$

Even if the $|b_n \rangle$ were expressed in some other basis, the above summation would still be the identity matrix. Now for any vector $|c \rangle$ we can construct for each $n$:

$$|b_n \rangle \langle b_n | c \rangle$$

The $|b_n \rangle \langle b_n|$ operator is called a projection operator, because it projects its argument onto the subspace spanned by $|b_n \rangle$, resulting in a component vector of the argument in the direction of $|b_n \rangle$. Clearly if we act with the same projection operator again on that result, nothing will change, because it's already projected. This is a way of defining a projection operator: it's idempotent.

And the sum of all those resulting vectors for all $n$ will just be $|c \rangle$, of course, because we've done the equivalent of acting with the identity operator.

\section{Hilbert Spaces}

The vector spaces used to represent physical states are examples of Hilbert spaces, which for our purposes means that:

\begin{itemize}
  \item they have an inner product that we can use to get a non-negative real number for the modulus of a vector: $\sqrt{\langle a|a \rangle}$, and 
  \item they may be infinite dimensional.
\end{itemize}

The latter possibility includes infinities that are continuous (uncountable). Such vectors cannot be represented by a column of discrete values, not even an infinitely long column. Instead we have to specify a complex-valued function over a continuous (real) variable. Such functions can be added and scaled, as is required of a vector (§\ref{ch:linearity}) and so they qualify as elements of a vector space (§\ref{sec:vectors-space}) and we therefore have no choice but to admit that they are vectors.

The real value is analogous to the integer index that labels the rows in a column vector; instead of fetching the $i$th component by its position in the column, we evaluate the function with some real value $x$.

Similarly, whereas the inner product over discrete components is:

$$
\langle a | b \rangle
=
\sum_i
x_i^* y_i
$$

the inner product over functions $f$ and $g$ of a real variable $x$ is:

$$
\langle f | g \rangle
=
\int_{-\infty}^{+\infty}
f(x)^* g(x)
dx
$$

This is also called the overlap integral, because it measures the extent to which the two functions overlap, but it is most definitely also the inner product between two vectors. Thus we can in some sense find the square of the "length" of a function: $\langle f | f \rangle$. This sounds like gibberish, but it is an unavoidable consequence of the definition of a vector space, which is abstract enough to admit a space of possible functions.

\section{Physical Interpretation}

To interpret the state vector physically, we choose a basis so we can resolve it into components. Our choice of basis has to do with the observable quantity we are presently interested in, such as position, momentum, orientation or energy. If it may take on any real value, the state vector will have to be a function of that value; if it may only take on certain discrete values, it can be a column vector (albeit sometimes one with infinitely many rows) in which each row corresponds to one of those possible discrete values that the observable may exhibit when measured.

The information available from the state vector is, in general, probabilistic. Each component, being a complex number, is related to the probability of the observable quantity taking on the value represented by that component. The squared modulus of the component (its value multiplied by the complex conjugate of its value) is the probability of obtaining that value, or if the state vector is a function $f(x)$, then:

$$\int_{a}^{b} f(x)^* f(x) \, dx$$

is the probability that $x$ will have a value somewhere between $a$ and $b$.

As a probability is a number between $0$ and $1$, it must be the case that the sum of the squared modulus of all the components (or the above integral from $-\infty$ to $+\infty$) must be $1$. This is the same as saying that $\langle S | S \rangle = 1$ for any physically realistic state vector. Or to put it another way, the magnitude of a state vector is not significant, only the direction (i.e. the relative values of the components in some basis). We will always fix the magnitude to be $1$.

Unsurprisingly, if one of the components is $1$ and all the others are zero, the vector represents certainty that the observable has the value represented by that component. But this also means that the state vector is equal to one of the basis vectors. Thus the basis vectors for an observable represent exact values that the observable may exhibit when measured.

Further, a measurement of the observable (or more precisely, any interaction producing subsequent behaviour that could be used to infer the value of the observable) causes the state vector to change to the basis vector of that observable corresponding to the measured value. This change is (at least in this theory) assumed to be instantaneous and to have no mechanism that we can deduce anything further about.

Thus after measuring an observable, subsequent measurements of the same observable will with certainty produce the same result.

(This is not quite true in the continuous cases when the state vector is actually a function of a real variable. We don't expect to ever find such a system precisely aligned with a single base state, but instead to have at least some small spread of probabilities.)

\section{Switching Basis}

Having constructed a column representation of a state vector in one basis, relating to one observable, we can switch to another. The operation for doing this will depend on both the "before" and "after" bases (§\ref{sec:vectors-change-basis}). A state vector contains everything knowable about a system, including all we can know about any of its observable quantities. By re-expressing the same state vector as a different set of components in terms of the basis associated with a different observable, we recover the probability distribution for that observable.

As always when using an operator to transform a vector's components we need to be clear on whether we want to get a different vector in the same basis or the same vector in a different basis. In this case, physically we're talking about the latter; a state vector represents something physically real, and we're just changing how we describe it. On the other hand, mathematically all we have is the description, and the choice of basis is not entirely arbitrary because a basis relates to an observable quantity.

\section{Operators Representing Observables} \label{sec:qm-operators1}

Observables have an associated operator. Note that this is \textit{not} the same as the operator for converting a state vector to a different basis. In QM when we talk about the observable's associated operator, we are talking about something that is not directly of any use for converting between bases, though it will indicate how we could perform such an operation.

An observable operator can be applied to a state vector as a kind of test, but it is much more powerful when we picture it applying to every possible state vector (that is, all unit vectors in the space) to find out how it affects them.

In QM operators associated with observables are Hermitian or self-adjoint, meaning that for an operator $\hat{O}$:

$$\langle a|\hat{O} b \rangle = \langle \hat{O} a| b \rangle$$

This has a few useful implications:

\begin{itemize}
  \item in the discrete finite vector case, operators can be represented as a matrix $O$, $O^{\dagger} = O$, or $O_{ij} = O_{ji}^*$, so the main diagonal elements are real,
  \item regardless of representation, eigenvectors (§\ref{sec:vectors-eigen}) with distinct eigenvalues are orthogonal and complete (they span the space, so you can take a unit vector in each of these orthogonal directions and you have an orthonormal basis) and
  \item regardless of representation, their eigenvalues are real.
\end{itemize}

Think of the analogy of a Euclidean real plane vector space, and a symmetric $2 \times 2$ matrix $M$ operating on it. The eigenvectors are lines in the plane along which vectors do not change direction, only magnitude, when the operator is applied. Because the matrix is symmetric ($M_{ij} = M_{ji}$) these lines are orthogonal. So it is with an Hermitian operator in a complex space, with only the added complication of needing to be careful about taking the complex conjugate when comparing diagonally opposite elements.

The basis vectors of the observable are just unit vectors that are eigenvectors of the operator. That is, if you apply the observable's operator to every possible state vector, a subset of them will be scaled (by a potentially complex factor) without any change to their alignment. There will be a set of orthogonal unit vectors that pass this "alignment preserving" test, and these form the basis of the observable.

In other words, quantum mechanics is substantially about:

\begin{itemize}
  \item defining the operator for an observable,
  \item solving the eigenvalue equation for that operator (that is, finding its eigenvectors and their associated eigenvalues)
  \item using the eigenvectors as a basis for representing state vectors,
  \item assuming that when the observable is measured, the state will snap into alignment with one of those eigenvectors,
  \item interpreting a coordinate in that basis as a complex amplitude whose mod-square is the probability that the state will align itself with that basis vector,
  \item interpreting the eigenvalue associated with the basis vector as the measured value of the observable (the eigenvalues of Hermitian operators are real numbers, fortunately.)
\end{itemize}

If a system's state vector matches one of these eigenvectors, then the system is already in an eigenstate and if the observable is measured, the result will with certainty be the eigenvalue associated with that eigenstate.

Otherwise, the state vector will be a linear combination of the eigenstates, and if the observable is measured and found to have a particular value, then the state vector will have instantaneously realigned itself with an eigenvector having that eigenvalue.

It is therefore very clearly the case that an observable's operator should not be mistaken for a means to transform a state vector into the basis of the observable. Consider a state vector that is already aligned with an eigenstate of the operator, but is currently expressed as a linear combination of some other basis, so it has several non-zero coordinates. But in the basis of the observable, its coordinates should all be zero except for the eigenstate's coordinate, which will have some complex value of modulus $1$. The observable's operator clearly cannot bring about this transformation: the state vector's coordinates will all be scaled by the same factor (the eigenvalue), by the very definition of eigenvectors.

But if we solve the eigenvalue equation for the operator, we will know the complete basis of the observable, along with the eigenvalue associated with each basis vector, and we can then resolve our state vector against those basis vectors. Then we will have a set of coordinates that serve as probability amplitudes for the associated (measurable) eigenvalues.

In addition, there is a meaningful interpretation for the result of applying the operator for an observable to a given state vector: the inner product of that vector with the original state vector gives the expectation value (§\ref{ch:expectation}) of the observable.

While we've discussed all this in terms of more easily pictured finite-dimensional vectors with discrete complex components, all the same concepts translate to complex-valued functions of an integer or real parameter.

\section{The Wave Function}

One way to approach QM initially is to consider the position and momentum of an electron. These are continuous variables, so we will be working entirely with state vectors that are represented by functions of real variables, and operators that transform functions.

We model this situation as a continuous complex-valued function of position and time, $\Psi(x, y, z, t)$, very often abbreviated to $\Psi$. We will sometimes also consider functions only of space, $\psi$. (This upper/lowercase distinction is quite widespread but not universally observed.)

By considering only one spatial dimension we can picture the wave function at one instant as a line, somewhere along which the electron could be found. At each point $x$ on the line there is an associated complex plane (visualised as normal to the line), with an arrow lying in it, pointing out from the line. This is the complex value of $\Psi$ at that position $x$ and time $t$.

The complex plane should not be confused with vectors. Any given snapshot of $\Psi(x, t)$ at some instant $t$, given by a function $\psi(x)$, is itself an entire vector. The position $x$ labels a single infinitesimal component of the vector, and every such component is a complex number, which we can therefore visualise as a complex plane with an arrow on it.

So for example we could picture the arrows as making a corkscrew shape, rotating around the line such that the angle depends linearly on $x$, but the modulus of the complex value (the length of the arrow) happens to be constant in this example. This is the notional wave function for a free electron (no forces acting it) with a precisely defined momentum and therefore no defined position, something never observed in reality.

More generally, the arrow length will also vary with $x, t$. The arrow length at $x$ determines the likelihood that the electron will be found at $x$. More precisely, the modulus-squared of $\Psi$, which can be calculated with $\Psi^*\Psi$, is proportional to the probability density:

\begin{equation}
  \rho(x) = \Psi^*\Psi
  \label{eqn:pdf}
\end{equation}

Given the electron is in some region $A$ between $x_1$ and $x_2$, the integral:

$$
\alpha =
\int_{x_1}^{x_2}
\Psi^*\Psi
\,dx
$$

is \textit{proportional} to the probability of finding the electron in $A$.

Recall that the product of a complex number and its own complex conjugate is a real number, and here we are doing $\Psi(x)^*\Psi(x)$, using the single complex value at position $x$, so the result will be real. But the complex conjugate is not a general purpose magic way to get a real number from a product of any two complex numbers; $\Psi(x_1)^*\Psi(x_2)$ need not be real.

If we compute the same integral $\beta$ for some larger surrounding region $B$, we can compute the conditional probability:

$$
P(A|B) = \frac{\alpha}{\beta}
$$

That is: the probability of finding the electron in $A$ \textit{given that} it is somewhere in $B$ is given by the fraction $\alpha / \beta$.

If $\Psi$ is suitably behaved (square-integrable; roughly, it goes to zero at some distance and does not become infinite anywhere) then we can compute the integral over the whole of our one dimension of space:

$$
\alpha =
\int_{-\infty}^{+\infty}
\Psi^*\Psi
\,dx
$$

We can then include a factor of $1/\sqrt{\alpha}$ within $\Psi$ to "normalise" it, such that integrating the normalised $\Psi^*\Psi$ over some region will directly give us the absolute (unconditional) probability of finding the electron in that region.

Some interesting things to note at this early stage:

\begin{itemize}
  \item For the simple first example of the free electron with definite momentum, normalisation is not possible because the integral over all of space does not converge on a finite value.
  \item A global change in the amplitude of the function (scaling the entire function by some complex constant) is not a physically significant change; there is a set of wave functions $a\Psi$ for any complex constant $a$, which all mean the same thing. What matters is how the amplitude varies from place to place (the same will turn out to be true for the complex phase).
  \item To normalise, we have to find the sum over all space of the mod-squared wave function. Interpreting the wave function as a vector, we're taking the inner product of the vector with itself, so we are in a sense finding the "length"-squared of the wave function as a vector. We then can then use this factor to scale it to be a unit vector, but preserving the relative shape of the wave (that is, preserving the "alignment" of the vector).
\end{itemize}

\section{Schrödinger Equation}

Any wave can be described as a sum of many simple component waves. (It is interesting that we use the word "component"; they are also basis vectors, so in vector terminology we should use the word component to refer to the complex constant factor applied to each simple wave included in the sum).

Each individual component wave has \textit{two} parameters:

\begin{itemize}
  \item if we nominate a fixed point in space, there is a frequency of oscillation, $\nu$
  \item if we freeze time, we can measure the wavelength, $\lambda$, the distance between adjacent peaks in space
\end{itemize}

These can be independently adjusted (do not be confused by the familiar example of EM waves, where wavelength and frequency are coupled due to the constant speed of light!)

So the component wave can be described by the complex exponential:

$$
\Psi(x, t) = \exp \left[ 2\pi i(\frac{x}{\lambda} - \nu t) \right]
$$

Pick any fixed point in space, so $x$ is constant, and $\nu$ determines the rate of oscillation. Pick a fixed instant in time, so $t$ is constant, and $\lambda$ determines the distance between peaks. With both in play, we have a corkscrew complex wave pattern that is moving.

Anything we figure out for this model wave can be taken to be true for any linear combination of many such waves, in the sense that we can imagine decomposing some messy wave into a set of components, each component characterised only by two numbers.

Planck inferred the relationship between frequency and energy:

$$\nu = \frac{E}{h}$$

And de Broglie likewise for momentum and wavelength:

$$\lambda = \frac{h}{p}$$

So we can write the wave function very neatly in terms of energy and momentum instead:

$$
\Psi(x, t) = \exp \left[ {\frac{i(px - Et)}{\hbar}} \right]
$$

Nothing much has changed: as before, we have two parameters shaping a complex corkscrew wave. (We use $\hbar = h/2\pi$ for brevity because that combination isn't going away.) All that has changed is that we've got two parameters with a physical interpretation for something we've previously thought of as a "particle".

We can take the partial differential of the above w.r.t $t$ or $x$, and the way that works with exponentials is strangely illuminating.

Doing $t$ first:

$$
\frac{\partial \Psi}{\partial t}
=
-\frac{iE}{\hbar}
\exp \left[ {\frac{i(px - Et)}{\hbar}} \right]
$$

The constant factor is copied outside the exponential, which otherwise remains the same. So in fact:

$$
\frac{\partial \Psi}{\partial t}
=
-\frac{iE}{\hbar}
\Psi
$$

We can tidy up by multiplying both sides by $i\hbar$:

$$
i\hbar \frac{\partial \Psi}{\partial t}
= E \Psi
$$

The exact same procedure with $x$ yields:

$$
- i\hbar \frac{\partial \Psi}{\partial x}
= p \Psi
$$

But we can also take the second derivative and get:

$$
- \hbar^2 \frac{\partial^2 \Psi}{\partial x^2}
= p^2 \Psi
$$

Returning to our physical interpretation, a free particle has energy that is purely kinetic, related to its momentum by:

$$
p^2 = 2m E
$$

(This is just $\frac{1}{2}mv^2$ smushed into the definition of momentum, $mv$.)

Substituting the Planck and de Broglie relations:

$$
\frac{\hbar}{2m} = \lambda^2\nu
$$

In general a corkscrew wave is governed by two independent parameters:

\begin{itemize}
  \item momentum, which goes with wavelength (and the $x$ coordinate)
  \item energy, which goes with frequency (and the $t$ coordinate)
\end{itemize}

We've now coupled them, making them no longer independent. But we've also added a new parameter: the particle's mass. For a free particle of a given mass, if you know the momentum you know the energy, and vice versa. Equivalently, if you know the wavelength you know the frequency, and vice versa.

Returning to the classical relationship between momentum, energy and mass, we can use it to rewrite our expression for $p^2 \Psi$, substituting into the R.H.S. to easily obtain:

$$
- \hbar^2 \frac{\partial^2 \Psi}{\partial x^2}
= 2mE\Psi
$$

And as we also have an expression for $E\Psi$, let's isolate that:

$$
E\Psi =
- \frac{\hbar^2}{2m} \frac{\partial^2 \Psi}{\partial x^2}
$$

and insert our $E\Psi$ expression:

$$
i\hbar \frac{\partial \Psi}{\partial t}
=
- \frac{\hbar^2}{2m} \frac{\partial^2 \Psi}{\partial x^2}
$$

So, recalling that $\Psi$ is an abbreviation for $\Psi(x, t)$, a complex valued function of space and time, now we have a differential equation that relates only these things:

\begin{itemize}
  \item $\hbar$, Planck's constant, a universal fixed real number with units of joules-seconds, very accurately determined by experiment, not something we can adjust to fit this equation to different scenarios
  \item $i$, which just provides a 90\textdegree phase shift
  \item the first partial derivative of $\Psi$ w.r.t. to time, which is another function of space and time that tells you how $\Psi$ is changing
  \item $m$, the mass of the particle
  \item the second partial derivative of $\Psi$ w.r.t. space.
\end{itemize}

This means that from a snapshot $\psi$ (at a specific instant of time) of the wave function of a particle with a known mass, so you have its shape in space, you can find the second derivative of that shape w.r.t. space, then multiply that by $i\hbar/2m$ and you have the the first partial derivative of $\Psi$ w.r.t. to time. That is, a snapshot contains complete information about the past and future of the wave; it tells you how to compute every past and future state.

So far, so kind-of rigorous. The situation becomes vaguer when we introduce a force field acting on the particle.

Schrödinger himself seems to have mostly taken a guess and found that the resulting equation agreed with several previously unexplained experimental results. Many widely used textbooks don't even give any background for it but merely state it. More advanced theory can be used to derive it, e.g. it is a low-energy approximation of QED.

The full classical account of the energy of a particle is:

$$
E = \frac{p^2}{2m} + V
$$

where the potential is a function $V(x)$. Realistically it will also be a function of $t$, but later we're going to pretend it isn't.

Some authors note that by multiplying the above throughout by $\Psi$:

$$
E\Psi = \frac{p^2{\Psi}}{2m} + V{\Psi}
$$

we obtain some scaffolding into which we can plug in our expressions for $E \Psi$ and $p^2 \Psi$:

\begin{equation}
i\hbar \frac{\partial \Psi}{\partial t}
=
- \frac{\hbar^2}{2m} \frac{\partial^2 \Psi}{\partial x^2}
+ V{\Psi}
\label{eqn:se}
\end{equation}

And this is the same as the free particle equation with the added $V\Psi$ term, and is the complete Schrödinger equation which governs the time evolution of $\Psi$.

The extra term doesn't change the important property that if you have a snapshot $\psi(x)$ taken of $\Psi(x, t)$ at a specific initial instant of time, then you know all future states (glossing over what happens when there is any kind of interaction, including measurements).

This is sometimes contrasted with Newton's 2nd law relating acceleration to force, acceleration being the second order derivative of the position w.r.t time. Each time we integrate we need to conjure up a constant of integration, and we have to integrate acceleration twice to get the position. The two constants we need to add are the position and velocity. Thus a snapshot of the position of a particle is not generally enough to know what is happening to it.

But a snapshot $\psi(x)$ taken of $\Psi(x, t)$ at some time is not just one number, but a continuous function giving a (complex) number at each point $x$ along the line, so it is generously endowed with information. If we decompose the snapshot into component waves, each one has its own wavelength.

And if we multiple $\Psi$ by some constant (possibly complex) factor, the result is still a solution to the function. Such arbitrary constant scale factors make no difference to the physical meaning; what matters is how the function varies from location to location (and from time to time). This is what allows us to normalise the function (where possible) to ensure that it sums to 1 over all of space.

\section{Time Evolution}

We can say little here about wave functions unless they can be normalised, i.e. wave functions that tend to zero at infinity. Assuming this is the case, if we integrate the PDF over all of space:

$$
\int_{-\infty}^{+\infty}
\Psi^*\Psi
\,dx
$$

we expect the result to be constant (if normalised, it should always remain 1 as time passes), i.e.

$$
\frac{d}{d t}
\int_{-\infty}^{+\infty}
\Psi^*\Psi
\,dx
= 0
$$

Note that as we are integrating over $x$, outside the integral $x$ is not a variable. We can move the differentiation w.r.t. $t$ inside the integral, but only we change it to partial, because inside the integral $x$ is a variable:

$$
\int_{-\infty}^{+\infty}
\frac{\partial}{\partial t}
\Psi^*\Psi
\,dx
= 0
$$

Focusing on the inside of the integral, by the product rule:

$$
\frac{\partial}{\partial t} \, \Psi^*\Psi
=
\frac{\partial \Psi^*}{\partial t} \Psi
+
\frac{\partial \Psi}{\partial t} \Psi^*
$$

Now, the Schrödinger equation gives us an expression for the partial time derivative of the wave function by slightly rearranging \eqref{eqn:se}:

$$
\frac{\partial \Psi}{\partial t}
=
\frac{i \hbar}{2m} \frac{\partial^2 \Psi}{\partial x^2}
- \frac{i V}{\hbar}{\Psi}
$$

From this we can get the same for the complex conjugate:

$$
\frac{\partial \Psi^*}{\partial t}
=
- \frac{i \hbar}{2m} \frac{\partial^2 \Psi^*}{\partial x^2}
+ \frac{i V}{\hbar}{\Psi^*}
$$

Plugging those into our expression:

$$
\frac{\partial}{\partial t} \, \Psi^*\Psi
=
\left[
- \frac{i \hbar}{2m} \frac{\partial^2 \Psi^*}{\partial x^2}
+ \frac{i V}{\hbar}\Psi^*
\right] \Psi
+
\left[
\frac{i \hbar}{2m} \frac{\partial^2 \Psi}{\partial x^2}
- \frac{i V}{\hbar}\Psi
\right] \Psi^*
$$

Multiplying out:

$$
\frac{\partial}{\partial t} \, \Psi^*\Psi
=
- \frac{i \hbar}{2m} \frac{\partial^2 \Psi^*}{\partial x^2}
\Psi
+ \frac{i V}{\hbar}\Psi^*\Psi
+
\frac{i \hbar}{2m} \frac{\partial^2 \Psi}{\partial x^2}
\Psi^*
- \frac{i V}{\hbar}\Psi\Psi^*
$$

The second and fourth terms cancel each other:

$$
\frac{\partial}{\partial t} \, \Psi^*\Psi
=
- \frac{i \hbar}{2m} \frac{\partial^2 \Psi^*}{\partial x^2}
\Psi
+
\frac{i \hbar}{2m} \frac{\partial^2 \Psi}{\partial x^2}
\Psi^*
$$

Also there's a common factor we can pull out:

$$
\frac{\partial}{\partial t} \, \Psi^*\Psi
=
\frac{i \hbar}{2m}
\left[
\frac{\partial^2 \Psi}{\partial x^2}\Psi^*
- \frac{\partial^2 \Psi^*}{\partial x^2}\Psi
\right]
$$

Recall that we are working out an expression for this because it appears inside an integral over all space:

$$
\int_{-\infty}^{+\infty}
\frac{i \hbar}{2m}
\left[
\frac{\partial^2 \Psi}{\partial x^2}\Psi^*
- \frac{\partial^2 \Psi^*}{\partial x^2}\Psi
\right]
dx
$$

Now the fundamental theorem of calculus is that integration is the inverse of differentiation, so there is clearly some redundancy here in that we are taking the second partial differential w.r.t. $x$ only to then integrate over all $x$.

To make this explicit:

\begin{equation}  
\frac{\partial}{\partial t} \, \Psi^*\Psi
=
\frac{i \hbar}{2m} \
\left[
\frac{\partial}{\partial x}
\left(
\frac{\partial \Psi}{\partial x}\Psi^*
- \frac{\partial \Psi^*}{\partial x}\Psi
\right)
\right]
\label{eqn:qm-byparts}
\end{equation}

The integral and the partial differentiation w.r.t. $x$ cancel out to give us an expression that we can evaluate at the two limits and take the difference:

$$
\frac{d}{d t}
\int_{-\infty}^{+\infty}
\Psi^*\Psi
\,dx
=
\frac{i \hbar}{2m}
\left[
\frac{\partial \Psi}{\partial x}\Psi^*
- \frac{\partial \Psi^*}{\partial x}\Psi
\right]
\bigg\rvert_{-\infty}^{+\infty}
$$

If we do that, we will have an expression for the rate of change, w.r.t. to time, of the integral of $\Psi^*\Psi$ over all space.

But at these limits, we've said $\Psi$ goes to zero, so as to be normalisable, making the whole expression zero at those limits. So in fact we've shown that, as we wanted:

$$
\frac{d}{d t}
\int_{-\infty}^{+\infty}
\Psi^*\Psi
\,dx
= 0
$$

So if it is possible to normalise a wave function at all, and it satisfies \eqref{eqn:se}, then the constant of normalisation lives up to its name: it is the same for all time.

\section{Motion}

Given this abstract notion of an electron being entirely represented by a complex-valued function of position, how can we make sense of an electron moving?

Supposing the wave function is more concentrated in some region, it makes sense to compute the expectation value of the position variable:

$$
\langle x \rangle =
\int_{-\infty}^{+\infty}
x \, \rho(x)
\,dx
$$

Substituting our definition of $\rho$ from \eqref{eqn:pdf}:

$$
\langle x \rangle =
\int_{-\infty}^{+\infty}
x \, \Psi^*\Psi
\,dx
$$

remembering always that $\Psi$ is short for $\Psi(x, t)$, so $\langle x \rangle$ is also a function of $t$, and so this gives us a way of thinking about motion: the way the expectation value of the position changes with time.

$$
\frac{d}{dt} \langle x \rangle =
\frac{d}{dt}
\int_{-\infty}^{+\infty}
x \, \Psi^*\Psi
\,dx
$$

We can rearrange to move the derivative inside the integral, giving:

$$
\frac{d}{dt} \langle x \rangle =
\int_{-\infty}^{+\infty}
x \frac{\partial}{\partial t}
\, \Psi^*\Psi
\,dx
$$

Like before, it's the $t$-derivative of something that depends on $x$, inside the integral over $x$ we clarify that it is the partial derivative, and therefore $x$ is a constant for that derivative.

And borrowing from \eqref{eqn:qm-byparts} we can rewrite this as:

$$
\frac{d}{dt} \langle x \rangle =
\frac{i \hbar}{2m}
\int_{-\infty}^{+\infty}
x
\frac{\partial}{\partial x} \
\left(
\frac{\partial \Psi}{\partial x}\Psi^*
- \frac{\partial \Psi^*}{\partial x}\Psi
\right)
\,dx
$$

This isn't as simple as before where we cancelled out the integration and the differentiation, because of the pesky $x$. But the good news is this is the easiest ever opportunity for integration by parts. Recall:

$$
\int
u
\frac{dv}{dx}
dx = uv -
\int
v
\frac{du}{dx}
dx
$$

So $u$ is just $x$ and to get $v$ we have to calculate it at the limits:

$$
v =
\frac{\partial \Psi}{\partial x}\Psi^*
- \frac{\partial \Psi^*}{\partial x}\Psi
\bigg\rvert_{-\infty}^{+\infty}
$$

Plugging them in:

$$
x
\left(
\frac{\partial \Psi}{\partial x}\Psi^*
- \frac{\partial \Psi^*}{\partial x}\Psi
\right)
\bigg\rvert_{-\infty}^{+\infty}
-
\int_{-\infty}^{+\infty}
\left(
\frac{\partial \Psi}{\partial x}\Psi^*
- \frac{\partial \Psi^*}{\partial x}\Psi
\right)
\frac{dx}{dx}
dx
$$

As before, with $\Psi$ vanishing at infinity the first term can be removed, and of course $dx/dx$ is $1$. Finally the above is just the integral from our $\langle x \rangle$ expression, so:

$$
\frac{d}{dt} \langle x \rangle = -
\frac{i \hbar}{2m}
\int_{-\infty}^{+\infty}
\left(
\frac{\partial \Psi}{\partial x}\Psi^*
- \frac{\partial \Psi^*}{\partial x}\Psi
\right)
dx
$$

Having unwrapped one layer with integration by parts we can pull the same trick with $\frac{\partial \Psi^*}{\partial x}\Psi$, with $u = \Psi$ and $v = \Psi^*$, which once again means the $uv$ term is zero, leaving:

$$
-
\int_{-\infty}^{+\infty}
\frac{\partial \Psi}{\partial x}
\Psi^*
$$

So putting this back into $\langle x \rangle$:

$$
\frac{d}{dt} \langle x \rangle = -
\frac{i \hbar}{2m}
\int_{-\infty}^{+\infty}
\left(
\frac{\partial \Psi}{\partial x}\Psi^*
+ \frac{\partial \Psi}{\partial x}\Psi^*
\right)
dx
$$

The two identical terms cancel with the $2$ on the bottom of the fraction, so:

$$
\frac{d}{dt} \langle x \rangle = -
\frac{i \hbar}{m}
\int_{-\infty}^{+\infty}
\frac{\partial \Psi}{\partial x}\Psi^*
dx
$$

If we think of the rate of change of $\langle x \rangle$ as the expectation value of the velocity, or $\langle v \rangle$, we can multiply by $m$ to get $\langle p \rangle$, which actually cancels the $m$.

$$
\langle p \rangle = -
i \hbar
\int_{-\infty}^{+\infty}
\frac{\partial \Psi}{\partial x}\Psi^*
dx
$$

\section{Operators Again} \label{sec:qm-operators2}

Another way to look at what we're doing here is discovering operators. To apply an operator $\hat{O}$ and get its expectation value $\langle O \rangle$, the recipe is:

$$
\langle O \rangle =
\int_{-\infty}^{+\infty}
\Psi^*
\hat{O}
\Psi
\,dx
$$

How does this relate to our previous discussion about observable operators (§\ref{sec:qm-operators1})? We said that the operator for an observable has orthogonal eigenvectors, with eigenvalues that are the value that will be measured. Our wave function at an instant in time $\psi(x)$ is a vector. To get a coordinate from that vector, we evaluate the function for some position $x$, and so the vector has a coordinate for every point in space. Therefore it is a vector expressed in the "position basis".

If the particle is very precisely localised, the function's value (the coordinates) will be zero everywhere except at that precise location. At the theoretical extreme, it will zero everywhere except at an infinitesimal single position (§\ref{sec:fourier-spike}). That is, it will be a basis vector in the position basis.

An observable operator has to scale its eigenvectors by the value that would be measured for a state equal to that eigenvector. That is exactly what happens if we multiply $\psi(x)$ by $x$: if it is a pure spike (a complex value of modulus $1$) at some position $x_1$, and zero everywhere else, the spike (and thus the whole vector) will be scaled by the value $x_1$. Whereas if it isn't a pure spike (not a position eigenvector), each non-zero value will be multiplied by a different value (its own position value), which will distort the shape of the function (or equivalently, change the "direction" of the vector).

We also mentioned in passing that if we apply an observable's operator to a specific state vector, we get an adjusted vector, and if we take the inner product between the original state vector and the adjusted vector, the resulting scalar value will be the expectation value of the observable. And that is exactly what we are doing in the above integral:

$$
\langle O \rangle =
\langle \Psi| \hat{O} | \Psi \rangle
$$

Because $\Psi$ is a function of $x$ and $t$, by integrating over all $x$ we get a function of time, telling us the evolving expectation value of whatever observable the operator represents. This "operator sandwich" pattern is intuitively sensible when we apply the position operator to a wave function of position, because this fits precisely with how we understand the expectation value to be computed: it is the sum of every possible value multiplied by its probability of occurring. $\hat{x}|\psi\rangle$ is just $x \psi(x)$. If we multiply that by $\psi(x)^*$ then it will be $x$ multiplied by the probability of measuring the position to be $x$; clearly then the integral over all space will be the expectation value $\langle x \rangle$.

So the position operator $\hat{x}$ is just $x$ itself:

$$
\langle x \rangle =
\int_{-\infty}^{+\infty}
\Psi^*
\hat{x}
\Psi
\,dx
=
\int_{-\infty}^{+\infty}
\Psi^*
x
\Psi
\,dx
$$

The momentum operator $\hat{p}$, which we discovered above by looking for the expectation value of momentum, is $-ih\frac{\partial}{\partial x}$:

$$
\langle p \rangle =
\int_{-\infty}^{+\infty}
\Psi^*
\hat{p}
\Psi
\,dx
=
\int_{-\infty}^{+\infty}
\Psi^*
(-ih\frac{\partial}{\partial x})
\Psi
\,dx
= -ih
\int_{-\infty}^{+\infty}
\Psi^*
\frac{\partial \Psi}{\partial x}
\,dx
$$

Compared to the position operator, it is somewhat less obvious what the momentum operator is doing to produce an expectation value for momentum. The intuitive process would be to sum (over all possible momenta) the product of each momentum and its probability of being measured. That would seem to require a wave function of momentum instead of position, and an integral over all momenta. And yet here we still have an integral over all positions, involving a wave function of position. \textit{We have not yet changed basis.}

It is perhaps easier to comprehend this intuitively through the analogy with regular vectors, and by remembering that vectors are basis independent objects (§\ref{sec:vectors-geometric}). The observable operator acts on the state vector, in general changing both its length and alignment (unless the state vector happens to already be an eigenstate of the observable, in which case only the length changes). But this action would be the same regardless of the basis we are working in, and so it isn't necessary to get hung up on that point. Likewise, projecting the operated-on vector back on to the original state vector, to produce a scalar expectation value, is a geometrical, basis-independent operation. The inner product depends on the relative lengths and the angle between the two vectors. As long as when we change basis we do so in a way that preserves the inner product (that is, by a \textit{unitary} operator), then the choice of basis is physically irrelevant. 

Therefore, in this recipe for the expectation value, the operator $\hat{O}$, the state vector $| \Psi \rangle$ and the adjusted vector $\hat{O} | \Psi \rangle$ should all be understood as having an independent existence from any choice of basis:

$$
\langle \Psi| \hat{O} | \Psi \rangle
$$

That expression is a scalar value, and is the same regardless of the basis we work in. When we want to calculate it, we use the same basis throughout.

On the other hand, the expectation value is a meagre summary of the rich information available in state vector, so we shall retain our obsession with finding a way to change basis.

\section{Time Independent Potentials}

In the Schrödinger equation, if the potential $V$ is constant everywhere (and thus may as well be zero everywhere), it reduces to the free particle equation that fell out automatically from the fact that kinetic energy is tied to momentum. If you know the energy, you know the momentum and vice versa, which means that if you know the shape of a time-independent snapshot of the wave $\psi(x)$, then you know everything.

If the potential is a function it gets trickier. To understand the effect of varying $t$ and $x$ separately, we can suppose the existence of two functions $\psi(x)$ and $\phi(t)$ that when multiplied give us $\Psi(x, t)$.

It is not generally true that this is possible. Even something as simple as $\Psi(x, t) = x + t$ can't separated into a product of two functions of $x$ and $t$. It's obviously true that solutions to the zero-potential Schrödinger equation can be separated, simply because we obtained it from the assumption:

$$
\Psi(x, t) = \exp \left[ {\frac{i(px - Et)}{\hbar}} \right]
$$

which can easily be written as the product of two separate functions of $x$ and $t$:

$$
= \exp \left[ {\frac{ipx}{\hbar}} \right]
\exp \left[ {\frac{-iEt}{\hbar}} \right]
$$

But when a potential is included it transpires that we can only use separation of variables if the potential is only a function of $x$, not $t$. But we can still discover some useful things.

$$\Psi(x, t) = \psi(x) \phi(t)$$

Taking partials becomes ordinary differentiation, because the other factor is constant:

$$
\frac{\partial \Psi}{\partial t}
= \psi \frac{d \phi}{d t},
\frac{\partial^2 \Psi}{\partial x^2}
= \frac{d^2 \psi}{d x^2}  \phi
$$

So we just plug those into \eqref{eqn:se}:

$$
i\hbar
\psi \frac{d \phi}{d t}
=
- \frac{\hbar^2}{2m}
\frac{d^2 \psi}{d x^2}  \phi
+ V \psi \phi
$$

Dividing by $\psi \phi$:

$$
i\hbar
\frac{1}{\phi}
\frac{d \phi}{d t}
=
- \frac{\hbar^2}{2m}
\frac{d^2 \psi}{d x^2}
\frac{1}{\psi}
+ V
$$

To make this explicit, let's put the parameters on each function:

$$
i\hbar
\frac{1}{\phi(t)}
\frac{d \phi(t)}{d t}
=
- \frac{\hbar^2}{2m}
\frac{d^2 \psi(x)}{d x^2}
\frac{1}{\psi(x)}
+ V(x)
$$

The LHS only depends on $t$, the RHS only depends on $x$. This means if we hold $x$ constant, and therefore the RHS constant, this equation still holds even if we vary $t$! And of course vice versa. Which means both sides are equal to the same constant, which we will call $E$ for a good reason (spoilers!)

Equating the LHS with $E$:

$$
i\hbar
\frac{1}{\phi}
\frac{d \phi}{d t}
= E
\therefore
\frac{d \phi}{d t}
=
- \frac{Ei}{\hbar}
\phi
\therefore
\phi = e^{-iEt/\hbar}
$$

The RHS isn't so neat, but:

$$
- \frac{\hbar^2}{2m}
\frac{1}{\psi}
\frac{d^2 \psi}{d x^2}
+ V
=
E
\therefore
- \frac{\hbar^2}{2m}
\frac{d^2 \psi}{d x^2}
+ V\psi
=
E\psi
$$

Solutions for $\psi$ will depend on $V$ of course. But the whole wave function is therefore:

$$\Psi(x, t) = \psi(x) e^{-iEt/\hbar}$$

Why is this interesting? Because the more complicated space-sensitive part is frozen w.r.t. time, we can understand the time evolution by just looking at the extremely simple factor:

$$
e^{-iEt/\hbar}
$$

Whatever the solution to $\psi$, the complex value of every point in space is only changing by the above factor as time passes.

And that factor is really just $e^{i\theta}$ with the angle being $-Et/\hbar$, so we know the modulus of the value isn't changing; it's just going "round and round" clockwise in the complex plane.

And if the modulus isn't changing, the probability density isn't changing, so the particle isn't moving. Hence solutions of this type are known as \textit{stationary states}. The expectation value of the position is fixed, and so all other observables' expectation values are also constant, including energy.

When we constructed the Schrödinger equation \eqref{eqn:se} we did so by building an expression for the total energy in terms of the particle's kinetic energy and a potential:

$$
- \frac{\hbar^2}{2m} \frac{\partial^2 \Psi}{\partial x^2} + V{\Psi}
$$  

Comparing this to our RHS differential equation:

$$
- \frac{\hbar^2}{2m}
\frac{d^2 \psi}{d x^2}
+ V\psi
=
E\psi
$$

So in one of these stationary states, $\psi$ substitutes for $\Psi$, but otherwise it's the same. We can extract an operator for the energy (the Hamiltonian):

$$
\hat{H} = 
- \frac{\hbar^2}{2m} \frac{\partial^2}{\partial x^2} + V
$$

And in a stationary state, it's just:

$$
\hat{H}\psi
=
E\psi
$$

This is an eigenvalue equation: for some solutions $\psi$, the $\bar{H}$ operator has the same effect as multiplying by a constant, the energy eigenvalue $E$ for that $\psi$. Another example that fits the vector-based framework for quantum mechanics that we began with.

All this is only true for the stationary states of separable $\Psi(x, t)$ wave functions, but we can sum an infinite set of them to get other shapes:

$$
\Psi(x, t)
=
\sum_{n=1}^\infty
c_n
\psi_n
e^{-iE_nt/\hbar}
$$

So for each $n$ there's a complex constant $c_n$ to go with the stationary state $\psi_n$ and an energy level $E_n$ that controls how fast the global phase shift goes round and round.

Because we're adding complex values at each point in space, even though those component values each have a time-independent modulus, the sum of them does not. So this is a way to make non-stationary solutions. Wave packets that "move" can be composed by summing stationary states that do not. A particle confined inside a potential can be in a stationary state, or it can "slosh" from side to side in a complicated way due to being a linear combination of many such states.

\section{Wave Functions as Vectors}

For stationary states $m$ and $n$:

$$
\int
\psi_m^* \psi_n dx = \delta_{nm}
$$

This does not mean that $\psi_m^* \psi_n$ is zero everywhere if $m \ne n$, but it does mean that for every non-zero value pointing in some direction in the complex plane, there's another value of the same modulus pointing the opposite way, to balance it out.

The above is a way of defining the inner product between two stationary states, and of showing that they are like orthogonal vectors in a complex vector space for that product.

In fact we can define our vector space for QM in two ways:

\begin{itemize}
  \item as an uncountable continuum of \textit{scattering states}
  \item As a countable infinity of \textit{bound states}
\end{itemize}

\subsection{Scattering States}

In the uncountable case of scattering states, the vector is a function of a continuous coordinate and cannot be reduced to anything more compact than that. This happens when the potential is absent and we have a physical state that moves freely through space as a pulse waveform. The inner product is defined by an integral:

$$
\langle \alpha | \beta \rangle
=
\int
\alpha^* \beta \,dx
$$

If we knew a particle's exact location, $\alpha$, our wave function of space $\psi(x)$ would have a single spike where $x = \alpha$ and be zero everywhere else. Alternatively if knew its exact momentum (and $p=h/\lambda$) our wave function would be a wave with a single wavelength. So we're dealing with Fourier transforms. At these extremes of certainty/uncertainty, one domain has a simple wave of infinite extent, and the other domain has a spike representing that wave. It works either way round.

Thus far we've been working in "position space", using functions of $x$, but alternatively we could work in "momentum space", where the functions are $\phi(p)$. If we knew a particle's exact momentum, $\phi(p)$ would be a spike, whereas if we knew its exact position, $\phi(p)$ would be a single-component wave.

Either way, the point is that the elements of our vector space are functions of a continuous variable (a real number), and the inner product has to be an integral over that continuous variable.

A theoretical free particle, being a perfectly periodic wave that extends to infinity, although it is a superposition of all possible position eigenstates, is on the other hand a pure eigenstate of both momentum and energy.

\subsection{Bound States}

In the countable case of bound states, a potential traps a particle and the measurable energies are quantised, the energy being the eigenvalue of the energy operator applied to the stationary state eigenfunction at that energy level.

The energy eigenfunctions serve as a set of basis vectors. Why? Because the energy operator is Hermitian (the equivalent of a symmetric matrix except with complex elements) which means that among its eigenvectors, any two corresponding to distinct eigenvalues will be orthogonal, and so each possible energy value is represented by an eigenvector. We can create a weighted sum of them to make any possible state. We can use those weightings as the components of a vector describing a state. That is, the state of the lowest energy level is a column vector of numbers where the first component is $1$ and all the other components are $0$.

There are two complications to this story:

\begin{itemize}
  \item In some situations we find that multiple eigenvectors have the same eigenvalue, so we cannot infer the state vector from an energy measurement.
  \item We can't measure the energy with absolute precision anyway. No system ever really collapses so its state vector is known to be precisely one of the eigenvectors of whatever observable is being measured. It always remains a superposition to some extent.
\end{itemize}

The inner product of two vectors is the sum of the products of the components of the two vectors (taking care to always complex conjugate the first one):

$$
\langle \alpha | \beta \rangle
=
\sum_i
\alpha_i^* \beta_i
$$

So once we've established the basis and computed the eigenvalues, we can construct states that are combinations of the eigenvectors and figure out the probability of obtaining a given energy by summing, rather than integrating.

\section{Energy Degeneracy}

Given two eigenvectors of an Hermitian operator, if they have different eigenvalues then they are orthogonal (intuitively they can't be colinear because a linear operator can't apply a different scaling to vectors that are colinear). But what about the converse? If they are orthogonal do they have different eigenvalues?

Given two eigenstates of the energy operator, $\psi_1$ and $\psi_2$, when we apply the operator it just scales them each by their own eigenvalue:

$$\hat{E}\psi_1 = E_1\psi_1$$
$$\hat{E}\psi_2 = E_2\psi_2$$

If $E_1 \ne E_2$, the two states must be orthogonal. What if we sum the two states and apply the energy operator to the result? The operator is linear, so applying it to a sum of states is the same as applying it to the states separately and then summing the result:

$$\hat{E}(\psi_1 + \psi_2) = \hat{E}\psi_1 + \hat{E}\psi_2 = E_1\psi_1 + E_2\psi_2$$

So the result is a linear combination of the two states. If $E_1 \ne E_2$, this result is not   some single constant $E_3$ multiplied by the sum of the states:

$$\hat{E}(\psi_1 + \psi_2) \ne E_3 (\psi_1 + \psi_2)$$

But if $E_1 = E_2$, then we can pull out a single constant (let's use $E_1$):

$$\hat{E}(\psi_1 + \psi_2) = E_1 (\psi_1 + \psi_2)$$

In other words, any two eigenstates with the same eigenvalue can be added to get a third eigenstate. Obviously this is also true for any linear combination using weightings $a, b$, simply because if $\psi$ is an eigenstate then so is $a\psi$:

$$\hat{E}(a\psi_1 + b\psi_2) = \hat{E}a\psi_1 + \hat{E}b\psi_2 = E_1a\psi_1 + E_2b\psi_2 = E_1(a\psi_1 + b\psi_2)$$

So this is a subspace of the whole vector space, a set of eigenvectors with the same eigenvalue, known (inevitably) as an eigenspace. Any scalar multiple of an eigenvector is also an eigenvector, so the line along which an eigenvector lies is (though not often) called an eigendirection, and is the simplest eigenspace that can arise with a linear operator.

In the simple case of a symmetric matrix operating on the plane, it will pick out two orthogonal directions along which it will perform a pure scaling. It may stretch along one direction but squeeze the other. Any vectors colinear with these two directions will only be scaled, their directions unaffected, and thus are eigenvectors. All other vectors will be rotated, and so are not eigenvectors. The set of eigenvectors aligned with one of the directions will all have the same eigenvalue, and so they form an eigenspace.

In QM the magnitude of a state vector is not significant; we always normalise it to a unit vector anyway. Therefore it is truer to say that observable states are represented by eigendirections, and in a given direction we nominate the unit vector to be \textit{the} eigenstate in that direction.

But in the higher dimensional complex vector spaces of QM, some operators will have eigenspaces that are not mere lines, but are themselves multidimensional. That is, there will be a set of vectors that are eigenvectors sharing the same eigenvalue, among which we can find an orthonormal basis of 2 or more dimensions.

The operation of scaling equally in all directions is a trivial example where all vectors are eigenvectors, and all have the same eigenvalue (the scaling factor), so they are all in the same eigenspace, and we can select any N that are not colinear to use as a basis of the N-eigenspace. Suppose we're dealing with a three dimensional vector space, and we scale up by a factor of $1.5$ along the $x$ and $y$ directions, but shrink by $0.5$ along the $z$ direction. A vector aligned with either the $x$ or $y$ direction will be an eigenvector under this scaling operation, with the eigenvalue 1.5. So this is degeneracy: orthogonal eigenvectors with the same eigenvalues, sharing a planar eigenspace that is a proper subset of the whole space.

So while we can say confidently that if two eigenvectors of an Hermitian operator have different eigenvalues then they are orthogonal, we cannot claim the converse: two orthogonal eigenvectors do not necessarily have different eigenvalues. In some circumstances they do, but not all. In particular, by classical intuition, the system of a weight on a string swinging back and forth has a total energy related to the weight's maximum displacement, but it can swing along any axis, so there is an infinite set of states with the same energy. So it is in QM. The inability in some situations to determine the state from the energy is known as degeneracy.

\section{Fourier Transforms}

A function $\psi(x)$ can be constructed by summing a collection of simple component waves, each with a different amplitude $A$ and wavelength, although we replace the latter with momentum $p$. So one component is $Ae^{ipx}$, and we can suppose the existence of a function $\phi(p)$ that tells you the amplitude $A$ for a given momentum. Then if we integrate over all momenta:

$$\psi(x) = \int \phi(p)e^{ipx} dp$$

This tells you the total amplitude of the wave at a given position. But it's also structurally equation \eqref{eq:invfourier}, implying that we can do the reverse:

$$\phi(p) = \int \psi(x)e^{-ipx} dx$$

So we can flip between "position space" and "momentum space" descriptions of the same state without loss of information. We've got two different functions that describe that state, functions of different variables, with (in general) different shapes.

We've also said that in QM we apply operators to switch between vector representations of the same state, in order to get their complex components in the representation for the observable for which want a probability distribution. So how does this work with the momentum operator?

Given a state described by position, $\psi(x)$, we can apply the momentum operator:

$$
\hat{p}\psi(x) = -i\hbar \frac {\partial \psi(x)}{\partial x}
$$

Suppose $\psi(x)$ is a simple wave of amplitude $A$ (the amplitude isn't yet relevant, and we have no way of normalising such a function, but it will play a role in a moment):

$$
\psi(x) = Ae^{ipx/\hbar}
$$

To differentiate an exponential we just multiply it by whatever factor is applied inside the exponential, in this case $ip/\hbar$:

$$
\frac{\partial}{\partial x} \psi(x) = \frac{ip}{\hbar}Ae^{ipx/\hbar}
$$

So applying the momentum operator in full:

$$
\hat{p}\psi(x) = -i\hbar \frac{ip}{\hbar}Ae^{ipx/\hbar}
$$

The $\hbar$s cancel, the $i$s become $1$, so we're left with:

$$
\hat{p}\psi(x) = p Ae^{ipx/\hbar} = p \psi(x)
$$

So it's an eigenvalue equation: the observable quantity $p$ is the eigenvalue of the operator $\hat{p}$, and so therefore we know that our choice of $\psi(x)$, a corkscrew wave, was an eigenstate of the momentum operator (we've just confirmed that the momentum operator does its job). But while any $\psi(x)$ might not be an eigenstate of momentum in this way, it will always be a linear combination of eigenstates of momentum. We just need a function $\phi(p)$ that gives us the amplitude (replacing $A$) contributed by every possible momentum $p$, and then we can integrate over all possible momenta to build the position wave function:

$$
\psi(x) = \int \phi(p)e^{ipx/\hbar} dp
$$

Differentiation is linear: we can perform it on each component wave and sum the results and this is equivalent to differentiating the sum of the components. So the steps we went through above can be performed with any composite wave too. However, we will not find a simple eigenvalue.

To make this explicit, let's make our position wave function from just two different momentum eigenstates:

$$
\psi(x) = A_1e^{ip_1x/\hbar} + A_2e^{ip_2x/\hbar}
$$

Applying the momentum operator:

$$
\hat{p}\psi(x) = p_1A_1e^{ip_1x/\hbar} + p_2A_2e^{ip_2x/\hbar}
$$

The two terms of $\psi(x)$ have each been scaled by a different momentum value. Furthermore they have different wavelengths so will interfere messily. So evidently this is not an eigenvalue equation.

But the way we've defined $\psi(x)$, we know it has a certain spectrum: two spikes, at $p_1$ and $p_2$, and nothing at any other momenta. So as a function of $p$ we would say:

$$
\phi(p) = A_1\delta(p - p_1) + A_2\delta(p - p_2)
$$

Let's compare the above discussion with regular vectors. A vector $\vec{v}$ in one basis has a set of coordinates in that basis, $v_k$. The vector can be constructed by multiplying each basis vector ${\vec{e}_k}$ by the corresponding coordinate and summing:

$$
\vec{v} = \sum_k \vec{e}_k v_k
$$

But we can transform the coordinates $v_k$ by using an operator described by a matrix $O$. The coordinates, $u_j$, still describe the same vector but in a different basis, and are given by the matrix multiplication $Ov$, which means producing the $j$th coordinate in the new basis by taking the dot product of the $j$th row of the matrix with the coordinates in the original basis, so here $j$ is the index of the coordinate we are calculating and $k$ remains the summation variable:

$$
u_j = \sum_k O_{jk} v_k
$$

In Dirac notation we can write the inner product between two vectors as $\langle a | b \rangle$. Supposing we wanted to obtain the first coordinate of our vector $\vec{v}$, we could write that as $\langle 1 | v \rangle$. The L.H.S. is not the numeric value 1; we're just using that as a label for the basis vector ${\vec{e}_1}$.

If the vectors are functions of a real variable such as position, the inner product is the integral over all values of that variable. But the basis vectors are $\delta$ spikes, the integral will just pick out the value at that position, and so it's the equivalent of simply evaluating the function at that position:

$$
\langle x | \psi \rangle
=
\int_{-\infty}^{+\infty}
\delta(x - y) \psi(y)
dy
=
\psi(x)
$$

So we see a kind of equivalence between the following:

\begin{itemize}
  \item the index $k$ becomes the position $x$
  \item the column vector of coordinates $v_k$ becomes the wave function $\psi(x)$
  \item the basis vector $\vec{e}_k$ is equivalent to the $\delta$ function that gives us a spike we can position at any $x$, i.e. the position basis
  \item the index $j$ becomes the momentum $p$
  \item $u_j$ is the momentum function $\phi(p)$
\end{itemize}

The tricky part is relating the matrix $O$ to the position-space momentum operator $\hat{p}$.

Each row of a matrix can be viewed as a row vector, applied to the input column vector with the inner product to produce one coordinate of the output vector. For continuous functions, that would mean that we need to perform an integral to obtain one output "coordinate".

Using a matrix to convert to another basis means getting the coordinate for each basis vector in the new basis, which means taking inner product between the input vector and each basis vector. So each row of the matrix is just one basis vector of the new basis, but \textit{expressed in the old basis}.

So to do this when converting from position basis to momentum basis, we need the set of functions that are the basis vectors of momentum \textit{expressed in the position basis}, so then for any one such basis vector we can do the inner product between it and our input vector to find the "coordinate" (complex value) for some momentum.

As we've seen, the basis vector for momentum $p$ expressed in the position basis is $e^{-ipx}$, a corkscrew wave, so the required inner product is simply:

$$\phi(p) = \int \psi(x)e^{-ipx} dx$$



\section{Phase-space Formulation}

Phase space is a conceptual space with both space and momentum dimensions. In the simplest case of motion in one spatial dimension, phase space has two dimensions, one each of space and momentum. Classically a harmonic oscillator traces out an ellipse in phase space, as it oscillates between zero displacement with maximum momentum, and then maximum displacement with zero momentum, back and forth (or in phase space, round and round). The two dimensions are not independent, because one is the derivative (or gradient) of the other.

There is a rarely used phase-space formulation of QM in which we characterise a system by a real scalar function of phase space, that is, $f(x, p)$. This function of both position and momentum is an alternative way to capture all the information in a complex wave function of \textit{either} position or momentum. For a simple harmonic oscillator the form of the function is rigid and simply rotates in phase space as time passes, in a direct analogy with the classical time evolution.

\section{Spin}

Up to this point we've been mostly considering a specific application of the general model of QM, where the classical concepts of position and momentum, which are the building blocks for anything else we might measure, are represented as functions of a continuous variable. The function's value is a complex number that can be mod-squared to get the probability density of that variable. There is something intuitive about this in the space domain, which is why we start with that: it makes us think of an electron as being spread out through space, and having a density that varies. Only when some interaction occurs does it appear to be concentrated entirely at one point. But the momentum of the electron is modelled the same way, and this is not by itself very intuitive. It only becomes a little clearer when we realise that the space representation is a wave, and we can model waves as a sum of simple component waves with a set of frequencies. The momentum function is the Fourier transform of the position function.

But now we're going to consider intrinsic angular momentum. This is introduced to explain how electrons are deflected in a magnetic field, behaviour which is classically suggestive of the electron spinning around an axis, although that physical interpretation seems impossible because the electron's radius (if it is non-zero) is extremely small. Antipodal points on the surface of the electron would have to be moving relative to each other faster than the speed of light. Pauli suggested glossing over this question and just accepting that electrons have an intrinsic angular momentum that cannot be interpreted in some comfortable classical way.

