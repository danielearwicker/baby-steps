\chapter{Relativity} \label{ch:relativity}

Newtonian mechanics is based on the notion that the passage of time is universal, and objects have motions that determine how their positions change with the passage of time.

Einstein (and Minkowski) overturned this. Space and time are dimensions of a combined \textit{spacetime}. The orientations of the space and time axes are a matter of perspective.

Rather than a point particle in space that is in motion, picture a path through spacetime, made up of points called \textit{events}. The standard term for this is a \textit{world line}, but I'm going to call it an \textit{event path} (as it is not necessarily a line).

So in such a structure there is no motion at all; it is fixed and permanent.

Straight segments of an event path correspond to uniform (non-accelerated) motion. What we call acceleration is any curved portion of the path.

To assert that a particle is "at rest" during some straight segment of its event path is to choose to align the time axis with that segment. To assert that a particle is "in uniform motion" is to choose a time axis that is not aligned with the particle's path.

Any diagram of spacetime we draw, with a time axis and space axis, necessarily requires us to choose a specific alignment for the time axis, and thus a space axis that is (from the perspective of one at rest) orthogonal to it.

\section{Clock Arrays, Spacetime Grids}

Consider particles that only have one degree of freedom, i.e. they take positions along a line. Nominate an origin on the line. At the origin, place an emitter of a pulse of light, and on either side of it, stretching off to infinity, place probes such that they are spaced one light-second away from their immediate neighbours.

Each probe contains a digital clock that measures elapsed seconds, but which is initially paused so its value does not advance as time passes. Each probe's paused clock displays an elapsed time that is equal to the probe's distance from the origin in light-seconds (note: we intentionally say the distance, which is always positive, not the displacement, which would be negative on the left and positive on the right).

At the origin, our emitter has a paused clock showing zero. It simultaneously emits the pulse of light and starts its own clock. When each probe detects the light pulse arriving, it starts its own clock.

In this way, we create a line of evenly spaced clocks that are synchronised with the origin's clock. We could now (if we wanted to) discard any notion of the distinct identity of each clock and treat them as an array of indistinguishable synchronised clocks. But instead we will label each clock with its displacement from the origin, so clocks on the left of the origin have increasingly negative labels, and those on the right have increasingly positive labels.

A spacetime diagram of this construction would be an orthogonal grid.

\begin{itemize}
    \item The intersections of the grid represent events where a specific clock ticked forward to a new whole number of seconds
    \item Each vertical line connects all such events for a single clock, so a vertical line \textit{is} a clock, in that it is the event path of a clock.
    \item Each horizontal line connects the events where a clock ticks forward to show a higher number of whole seconds have elapsed.
\end{itemize}

On such a diagram we can imagine our perception of time passing as being represented by a horizontal "line of simultaneity" that sweeps up through the grid.

\section{Curved paths}

A rogue particle is now introduced. It is free to move left or right on the line. It can accelerate freely. It is depicted on our spacetime grid by a curved path. On its journey it visits several of our clocks, which are able to sense when it passes very nearby and make a note of the time (according to that clock) when such a close encounter occurs. Thus we can build up a record of the movement of the particle, consisting of pairs of position (the clock's label) and time (the clock's time) captured at each clock it passes.

Because the rogue particle is outnumbered, it appears very clear that the the rogue is in non-uniform motion against the background of our original array of probes.

To even the score, suppose that rather than one such rogue particle, we have an array of them, spaced out along the line, and each having its own clock. These rogue clocks are synchronised by a light pulse just as before.

All the particles in the rogue array have been programmed to carry out a predetermined sequence of accelerations, by firing little rocket thrusters. They all perform these accelerations in perfect unison and so remain the same distance from each of their neighbours. They are another co-moving array.

The result is that relative to each other, the rogue clocks are seemingly at rest, and could presume that it is our original array of clocks that are doing all the accelerating, and it being mere chance that their own rocket firings coincide perfectly with accelerations undertaken by other particles.

But nature is not fooled. At some basic level, the act of accelerating is accounted for and is an objective fact, not something that can be defined away by a change of perspective. The straight line segments of an event path are objectively straight, and the curved segments are objectively curved. The only valid conversions between points of view must be linear, in that they never convert a straight path into a curved path or vice versa.

So we have two fundamentally different arrays of co-moving particles, one in uniform motion and the other in accelerated motion.

Even so, we can imagine following the path of one rogue particle across spacetime as its own clock ticks. Naively we might imagine rotating the diagram against our coordinate system as we trace the path, so as to keep the tangent to the path aligned with the time axis. After all, rotations are linear transformations that conserve the distance between each point and the centre of rotation. Each point moves in a circle around that centre.

The invariant separation between points is captured by the Pythagorean formula:

$$r^2 = t^2 + d^2$$

But the correct form of rotation to use between a space and a time coordinate is hyperbolic rotation, such that each point moves on a hyperbola as we rotate our perspective, and the equivalent of a distance between events being conserved is called the \textit{interval}, $s$:

$$s^2 = t^2 - d^2$$

It transpires that the interval between events is an objectively real fact that all observers agree on. The absolute structure of the universe is \textit{spacetime}, which is the set of all events, and all observers agree on the interval between any two events.

\section{Spacetime Metric}

We can define a metric, a way of taking the inner product of two vectors, and to find the squared length of a vector we take the inner product of the vector with itself. The metric for Euclidean geometry is the Kronecker delta, $\delta$, e.g. for three space dimensions:

$$
\delta =
\begin{bmatrix}
1 & 0 & 0 \\
0 & 1 & 0 \\
0 & 0 & 1
\end{bmatrix}
$$

According to that metric, the squared length of vector $\vec{v}$ in terms of its coordinates $v^i$ is:

$$|\vec{v}|^2 = \sum_i \sum_j \delta_{ij}v_iv_j$$

As $\delta$ picks out the terms where $i = j$:

$$|\vec{v}|^2 = (v_1)^2 + (v_2)^2 + (v_3)^2$$

This is the familiar theorem of Pythagorus.

But to find the interval between two events in spacetime, we need to use a different metric, $\eta$ (eta). Now our indices $i, j$ can take four values, traditionally given as $0, 1, 2, 3$, with time being $0$. So with a time dimension in addition to the three space dimensions, the correct metric happens to be:

$$
\eta =
\begin{bmatrix}
c & 0 & 0 & 0 \\
0 & -1 & 0 & 0 \\
0 & 0 & -1 & 0 \\
0 & 0 & 0 & -1 \\
\end{bmatrix}
$$

where $c$ is the speed of light, a quantity with cosmic significance. In honour of this we can take as our unit of distance the light-second, so that $c$ is 1.

> Note that the signature of the diagonal of $\eta$ could just as well be $(-, +, +, +)$ instead of $(+, -, -, -)$, and as with anything that makes no difference, debate has raged on for over a century.

Furthermore, as our particles only move along a straight line we only need one space dimension, so altogether the metric can be written as:

$$
\eta =
\begin{bmatrix}
1 & 0 \\
0 & -1 \\
\end{bmatrix}
$$

So we have arrive at the modified form of Pythagorus such that the interval, $s$ between two events separated by a distance $d$ and a time $t$ is given by:

$$s^2 = t^2 - d^2$$

It would be quite misleading to continue to picture this as a right triangle with $d$ and $t$ as the orthogonal short sides and $s$ as the hypotenuse, because $s$ has to remain constant as $d$ and $t$ both increase, which is impossible if the $d$ and $t$ sides remain orthogonal.

\section{Interval Related To Proper time}

A clock that travels inertially between two events separated by interval $s$ will measure the elapsed proper time $\tau$, but this is in fact identical to the interval:

$$\tau = s = \sqrt{t^2 - d^2}$$

This is obvious given that in its own coordinate system, it remains at coordinate zero, and so $d = 0$ and hence:

$$\tau = s = t$$

This underscores the invariant, universal, unambiguous nature of the interval between two events. Two synchronised clocks departing from some event, and arriving at another via different "routes" through spacetime, may each show a different elapsed proper time, and each's proper time is a measure of the "length" of the path it took (the sum of all the infinitesimal segments of interval along that path.)

But due to the nature of the metric, with time and space coordinates making opposite-signed contributions to the sum, less straight paths require \textit{less} time to elapse for the particle taking them. Thus the straight line path between the events (that which would be taken by a uniform motion) is the \textit{slowest} path possible, in that a particle taking that path will see its own clock advance by a greater duration than that of any particle taking some otherwise curved path.

\section{Lorentz Factor}

Consider a clock-carrying particle $P$ that is in uniform motion relative to an array of co-moving clocks, $A_n$. Two events are of significance:

\begin{itemize}
    \item $P$ passes by $A_0$ when both $P$'s and $A_0$'s clocks read $t_0$
    \item $P$ passes by $A_1$ when $A_1$'s clock reads $t_1$.
\end{itemize}

As measured in the $A$ frame, the elapsed time $t$ between the two events is:

$$t = t_1 - t_0$$

Also in $A$ the distance between the two events is the distance between the clocks:

$$d = A_1 - A_0$$

And with these two values, $A$ can compute the constant velocity of $P$ during its journey:

$$v = d/t$$

But things look a little different from the perspective of $P$. A uniformly moving clock regards itself as being at rest, so the interval between the two events is accounted for entirely by the passing of time, and so the elapsed time according to $P$ is equal to the interval of the path it takes.

We know that the interval is an invariant, objective fact about reality, not something that changes based on perspective, and we also know that it equals the time measured by a clock moving inertially between events. So using the values for $d$ and $t$ obtained in $A$ we can correctly calculate the elapsed time measured by $P$:

$$\tau^2 = t^2 - d^2$$

Pulling out a factor of $t^2$:

$$\tau^2 = t^2(1 - \frac{d^2}{t^2})$$

We can abbreviate this by using $A$'s value for the velocity $v = d/t$:

$$\tau^2 = t^2(1 - v^2)$$

And unsquaring both sides:

$$\tau = t\sqrt{1 - v^2}$$

So the ratio that will convert the proper time $\tau$ measured by $P$ back to the time coordinate separation $t$ as measured in $A$ is:

$$\gamma = \frac{1}{\sqrt{1 - v^2}}$$

This $\gamma$ is the Lorentz factor. Note that as $v$ is based on a time value $t$ in seconds and a distance value $d$ in light-seconds, it is a fraction of the speed of light.

If it had the value 1, that would be light speed. But at that precise value, $\gamma$ is undefined, due to the zero on the bottom of the fraction. The elapsed time for any journey taken by a photon is always zero, and so there is no way to relate times recorded by a photon to times recorded by an array of clocks (a photon is not a clock).

Furthermore, at velocities greater than 1 (faster than light), the value is defined but unfortunately is imaginary, due the square root of a negative value. The square of the interval between events travelled between by faster-than-light particles is negative, and so the interval, which is the elapsed time recorded by a clock carried by such a particle, is positive-imaginary.

\section{Lorentz Transformations}

Two observers $\rho$ and $\phi$ moving apart at speed $v$ will label the same event with different spacetime coordinates: $(t_\rho, x_\rho)$ and $(t_\phi, x_\phi)$.

To transform back and forth between these systems we need a pair of matrices, each the inverse of the other, such that these invariants are conserved:

\begin{itemize}
    \item the interval between events
    \item the speed of light
\end{itemize}

Note that for any

$$\begin{bmatrix}\gamma & v\gamma \\ v\gamma & \gamma\end{bmatrix}^{-1} = \begin{bmatrix}\gamma & -v\gamma \\ -v\gamma & \gamma\end{bmatrix}$$

\section{Energy and Momentum}

The momentum is a vector, so in 3 dimensions of space it is a 3-vector and can be resolved into 3 scalar components once a suitable basis has been chosen.

The kinetic energy is a scalar. But both are observer dependent. A particle at rest relative to the observer has momentum that is the zero vector and kinetic energy zero. Should another particle collide with the one at rest and cause both particles to travel away in new directions and speeds, the total momentum vector and the total kinetic energy scalar will be the same before and after the collision, but the values of these quantities are different depending on the observer.

To another inertial observer moving relative to this scene, exactly the same conservation laws will be found to be obeyed, just with different numbers involved. The two observers will disagree over the momenta and energies of the specific particles, and also over the total energy and momentum, as summed separately over both particles.

As before we will take as our unit of distance the light-second, so the speed of light $c$ is 1. In this unit system, energy and mass are fully equivalent, because the famous:

$$E = mc^2$$

becomes:

$$E = m$$

The total energy of a particle is its mass times the Lorentz factor $\gamma$:

$$\gamma = \frac{1}{\sqrt{1-v^2}} $$

In our everyday experience, $v$ is practically zero, as we're expressing it as a fraction of the speed of light, so almost all the energy of a particle is in its mass. The contribution from the kinetic energy is almost non-existent.

But again, this is a frame-dependent quantity, because a particle only has a defined velocity relative to some chosen inertial frame.

If we combine the components of the momentum 3-vector with the energy scalar (the total energy $m\gamma$ discussed above), we get a 4-vector called the 4-momentum. As always in the Minkowski metric, the magnitudes of these objects are related by:

$$m^2 = E^2 - |\vec{p}|^2$$

So the 4-momentum has magnitude $m$, the energy is "temporal" and the 3-momentum is "spatial". Enjoy the symmetry with:

$$s^2 = t^2 - d^2$$

It's interesting that the momentum provides the three spatial components, while energy provides the remaining temporal component (and indeed this is the case: momentum conservation is due to translational symmetry in space, and energy conservation is due to translational symmetry in time.)

As we know that momentum and energy are separately conserved from the viewpoint of any inertial observer, we therefore know that the combined 4-momentum must also be conserved.

The four coordinates are resolved relative to a coordinate system. We had to choose an orientation for the three axes that make up our spatial basis, which is a slice of spacetime. One way to visualise it is to discard one of the space dimensions, so that space is a planar slice through in a 3D spacetime. A given inertial observer regards their spatial slice through spacetime as containing all the events happening "now", making it a "slice of simultaneity".

Each inertial observer will use different coordinates for the 4-momentum, not just because they have a free choice of spatial basis, but also because they each have an event path through spacetime that is momentarily in a specific direction. Each observer, assuming themselves to be at rest (at least instantaneously), regards their own event path as aligned with the time axis, and orthogonal to their slice of simultaneity.

But despite the coordinates of the 4-momentum being different, they describe the same vector in spacetime. That is, two observers stating the 4-momentum of the same particle will use different coordinates for the same 4-momentum. The vector itself can now at last be said to be conserved even under a change of coordinate systems. Everyone agrees on what the 4-momentum is geometrically, so we no longer have to qualify the law of conservation of momentum with caveats about a single frame of reference.

If we scale the 4-momentum by $1/m$ (that is, in some coordinate system, if we divide all the components of the 4-momentum by the particle's intrinsic mass), we obtain the 4-velocity:

$$\vec{u} = \frac{\vec{p}}{m}$$

The magnitude of this vector is always $c$ (or $1$ in our simplified units), because the 4-momentum's magnitude is always $mc$ (or $m$). In other words, the only information really carried by the 4-velocity is a direction. We're only modelling a \textit{direction} in spacetime and can ignore the magnitude as not physically significant.

This division by $m$ is not meaningful for a massless particle such as a photon, which is why momentum is more fundamental than velocity, as momentum can be discussed for all particles regardless of whether they have mass. Yet it's interesting that a photon's 4-vector nevertheless has magnitude $c$ (or $1$), just as if it had a very small mass $m$ that we could divide by.

The relationship between the components of the 4-velocity $\vec{u}$ and familiar concepts is not as straightforward as for the 4-momentum. The spatial slice of $\vec{u}$ points in the direction of motion (of course), as does the ordinary velocity $\vec{v}$. But their magnitudes, $u$ and $v$, are different:

$$u = \gamma v$$

Where $\gamma$ is the Lorentz factor again. As we noted above, there's an identical relationship between the total energy $E$ and the intrinsic mass:

$$E = \gamma m$$
