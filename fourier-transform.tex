\chapter{Fourier Transform} \label{ch:fourier}

Given a real-valued function $f(x)$, and supposing it is periodic, e.g. it describes the sound of a bell ringing, you might ask what frequencies appear in the sound. In fact your ear-brain system is an adaptation for answering that very question, and if you listen carefully you can often discern several different notes within the sound of a bell.

What we're really asking is how "loud" the signal is at each frequency. We can detect this for a given frequency $\nu$ by multiplying the function by $e^{-i2\pi\nu x}$, in which:

\begin{itemize} 
    \item the minus sign is purely a convention (and not a universal one)
    \item $i$ is the magic ingredient that makes it go round and round
    \item $2\pi$ converts to radians
    \item $\nu$ is the frequency
    \item $x$ is the parameter to the function
\end{itemize}

So if $\nu$ is 1, the complex value performs a whole rotation as $x$ goes from $0$ to $1$, and again from $1$ to $2$, etc.

By itself this factor is a unit complex number, i.e. of "length" 1, but by multiplying it by the function we adjust its length so it oscillates "in and out" as it rotates, exactly like our signal:

$$f(x)e^{-i2\pi\nu x}$$

If the oscillations of $f$ don't coincide with the frequency $\nu$, the above expression will, averaged over all values of $x$, be about zero, there being no particular reason for the complex value to be biased in any direction. That is:

$$\int_{-\infty}^{\infty} f(x)e^{-i2\pi\nu x} dx \approx 0 $$

But if the oscillations do coincide, then there will be a bias; each time the oscillation of $f(x)$ reaches a maximum it will be on the same side of the circle traced by $e^{-i2\pi\nu x}$.

(A minor subtlety is that whenever $f(x)$ is at a negative minimum, $e^{-i2\pi\nu x}$ will be on the other side of the circle; however, multiplying it by the negative value of $f(x)$ will flip it round by $180$ degrees, so both positive peaks and negative troughs will both contribute to the same biased direction.)

So we can define a complex-valued function of frequency:

\begin{equation}
    \hat{f}(\nu) = \int_{-\infty}^{\infty} f(x)e^{-i2\pi\nu x} dx
    \label{eq:fourier}
\end{equation}

and this will be about zero for frequencies that don't appear in the function, and non-zero for frequencies that do appear. These values are \textit{complex} amplitudes; they tell us how loud the signal is at that frequency, but also their phase tells us how the signal is offset at that frequency.

As a shorthand we can write it as a fancy $\mathcal{F}$:

$$\hat{g} = \mathcal{F} g$$

We can do the opposite transformation:

\begin{equation}    
f(x) = \int_{-\infty}^{\infty} \hat{f}(\nu)e^{i2\pi\nu x} d\nu
\label{eq:invfourier}
\end{equation}

Shorthand:

$$g = \mathcal{F}^{-1} \hat{g}$$

This pretty much literally says that you can make any function by adding together an infinite collection of oscillations at every possible frequency. You just need a (complex) function $\hat{f}(\nu)$ that tells you how "loud" each frequency needs to be.

\section{Negative Frequencies}

By the way, note how when we do the integral over $\hat{f}$ in the inverse transform, we include negative values. What on earth is a negative frequency?! It's not that weird, really. It just makes the complex factor rotate the other way. This underscores the fact that the minus sign is just a convention. The integrals cover both "directions".

There's a special relationship between the positive and negative sides of the frequency spectrum. According to equation \eqref{eq:fourier}, if you want the amplitude for frequency $-\nu$, it must be:

$$
\hat{f}(-\nu) = \int f(x)e^{i2\pi\nu x} dx
\label{3}
$$

Which means $\hat{f}(-\nu)$ is the complex conjugate of $\hat{f}(\nu)$.

$$
\hat{f}(-\nu) = \left[ \hat{f}(\nu) \right]^*
$$

Therefore having done the hard work of computing one side, it is very easy to get the other side - it contains no different information.

Also, note that as we're integrating over all $x$ from $-\infty$ to $\infty$, we can negate $x$ through the integral without changing the result. So this produces the exact same frequency amplitudes as equation \eqref{eq:fourier}:

$$\hat{f}(\nu) = \int f(-x)e^{i2\pi\nu x} dx$$

Suppose $f$ happens to be an even function:

$$f(-x) = f(x)$$

Then we can switch freely:

$$\hat{f}(\nu) = \int f(x)e^{i2\pi\nu x} dx$$

If we mirror the frequency:

$$\hat{f}(-\nu) = \int f(x)e^{-i2\pi\nu x} dx$$

But we've arrived back at equation \eqref{eq:fourier}, meaning it must be perfectly symmetrical around $\nu = 0$ when applied to an even function. This means it must also be real at all frequencies - how else could all this be true?

$$
\hat{f}(-\nu) = \hat{f}(\nu) = \left[ \hat{f}(\nu) \right]^*
$$

Now suppose $f$ is odd:

$$f(-x) = -f(x)$$

By a similar argument, when we substitute:

$$
\hat{f}(\nu) = - \int f(x)e^{i2\pi\nu x}
dx
$$

And mirror:

$$
\hat{f}(-\nu) = - \int f(x)e^{-i2\pi\nu x}
dx
= -\hat{f}(\nu)
$$

If taking the complex conjugate is the same as negating, we must be talking about a purely imaginary number. So the transform of an odd function is imaginary and odd.

\section{Spikes} \label{sec:fourier-spike}

What happens if we take the Fourier transform of a pure $\sin$ wave? Only a single frequency is present. To describe this situation, the mathematical tool we need is called the Dirac delta, $\delta(x)$, and is often referred to as a function, or a "function" with scare-quotes. It has a number of strange properties if regarded as a function, so it's simpler to think of it as only ever appearing as a factor inside an integral. But in simple terms, it is zero except at $x = 0$, where it is infinite. We can use an expression like $\delta(x - \alpha)$ to move the spike from zero to the location $\alpha$ of our choice.\footnote{It's like the real number equivalent of the kronecker delta, though we write that slightly differently. You can think of the Kroneker $\delta_{nm}$ (§\ref{def:kronecker}) as conceptually similar to the Dirac $\delta(n - m)$: if $n \ne m$, the result is $0$.}

Why does it have to be infinite at the spike? We recover the function from its transform with the inverse transform, which is an integral:

$$
f(x) = \int_{-\infty}^{\infty} \hat{f}(\nu)e^{i2\pi\nu x} d\nu
$$

Substituting $\delta(\nu - \alpha)$ as the transform from which we're recovering the function, i.e. a spike at $\alpha$:

$$
f(x) = \int_{-\infty}^{\infty} \delta(\nu - \alpha)e^{i2\pi\nu x} d\nu
$$

Imagine the value of $\nu$ sweeping through the range of values from $-\infty$ to $+\infty$, everywhere contributing nothing except at the instant it passes through $\nu = \alpha$. That contributes $e^{i2\pi\alpha x}$. To accomplish that, the other factors must have the product $1$:

$$
\delta(\nu - \alpha) d\nu = 1
$$

and so:

$$
\delta(\nu - \alpha) = \frac{1}{d\nu}
$$

So the spike at $\alpha$ from $\delta(\nu - \alpha)$ must be infinite so as to counteract the infinitesimal smallness of $d\nu$. In other words, we have to think about any transform as an amplitude \textit{density} function.

\section{The Gaussian}

An infinitesimally narrow spike in the frequency spectrum represents the single frequency present in a pure wave that goes on forever. And a pure wave in the frequency spectrum represent the infinite set of frequencies that must be summed to get an infinitesimally narrow spike. Each is the Fourier transform of the other. They are the two extremes of:

- being localised in position but spread out in frequency
- being spread out in position but localised in frequency

Between these two there is a middle ground, a shape that is its own Fourier transform. The best known example is the Gaussian, of the form:

$$g(x) = Ae^{Bx^2}$$

Where $A$ and $B$ are constants. There are several ways of concluding that its Fourier transform is:

$$\hat{g}(\nu) = A \sqrt{\pi/B} e^{-\frac{\pi^2}{B} \nu^2}$$

This is evidently of the same form. Aside from being a function of frequency $\nu$ instead of position $x$, the constant $A$ has become $A \sqrt{\pi/B}$, and $B$ has become $-\frac{\pi^2}{B}$, and these are just different constants.
