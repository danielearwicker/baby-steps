\chapter{Potential}

A force field is a vector-valued function of space, i.e. at each point in space we imagine there is a vector giving the strength and direction of the force that would felt at that point.

The force fields we observe in nature have an interesting property: it is always possible to replace the force field with a scalar-valued function of space, i.e. at each point in space there is merely an ordinary number, not a vector. We can then take the vector gradient $\nabla$ of this scalar field and we recover the force field.

By analogy, picture a hilly landscape. The height $H$ above sea level is the scalar field value, so the landscape is fully described by the scalar field $H(x, y)$. From this we can derive $\nabla H$, a two-dimensional vector field (picture it as an arrow that never points up or down, always parallel to the horizon). As we travel around we sometimes face steep slopes, where $\nabla H$ points in the steepest direction, or stationary points such as hilltops or valley basins where $\nabla H$ is the zero vector (to distinguish between peaks and valleys, we'd need to take the second derivative, $\nabla^2H$).

If we wander on some pathway through this landscape and return back to where we started, our height will be the same as it was when we started (assuming the landscape hasn't changed shape). This is true regardless of the path we take, as the height is a fact about the start/end point of the path. This is so obvious as to seem hardly worth stating.

And yet if we only had some vector field, and wondered if the path integral of any closed loop through that field was always zero, how would we know? Some paths might go mainly through regions with vectors all pointing in one direction, and so not sum to zero. Not all vector fields have this self-balancing property.

Those that do are known as conservative fields, and these are fields which can be reduced to a scalar field from which the vectors can be recovered by applying $\nabla$, and these are all the force fields we encounter in nature.

When we describe a force field by a scalar field, we call that field a \textit{potential}. It has units of energy. As a particle moves through a potential, it experiences a potential difference between two points. If this difference is negative, i.e. the potential energy drops between the two points, the particle gains kinetic energy (speeds up). This is exactly like a ball rolling down a slope; the potential energy is exactly equivalent to the height of the landscape.

If the potential does not vary, the gradient is zero. This is true regardless of the potential's constant value, which is like a constant of integration, i.e. a global increase in potential is physically meaningless.

An important example is a force field conforming to the inverse square law, so the force is proportional to $r^{-2}$ where $r$ is the distance from the origin of the force. The potential must therefore be proportional to $r^{-1}$, so that it has the required gradient (differentiation subtracts 1 from the power of a polynomial).
