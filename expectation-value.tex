\chapter{Expectation Value}

This unfortunate statistical term is used everywhere; unfortunate because it describes a value that we do not necessarily expect to ever measure, and even more unfortunate that it is often garbled into ``the expected value'', which may be entirely untrue. It is the expected \textit{mean} of a set of repeated measurement values.

For a set of discrete values taken by some integer variable $n$, the values may be $2, 3, 3, 3, 4, 4, 5$, which sums to $24$, and there are $7$ values, so the mean value is $3.42857...$ which is not an integer so clearly cannot be an expected value.

Looking at the list of values, we can tabulate them by giving the observed (``frequentist'') probability $P$ of each value (number of times it occurs divided by the size of the set of values):

\begin{center}
    \begin{tabular}{ c|c|c }
        $n$ & $P_n$ \\
        \hline
        $2$ & $1/7$  \\
        $3$ & $3/7$  \\
        $4$ & $2/7$  \\
        $5$ & $1/7$
    \end{tabular}
\end{center}

$P_n$ is zero for all $n$ except the above exceptions, where it is between zero and $1$, and of course all values of $P_n$ add up to $1$ because we fixed them to do that when we divided them all by $7$. $P_n$ is literally ``what fraction of the $7$ values is contributed by $n$''.

Therefore by computing the weighted sum:

$$ \langle {n} \rangle = \sum_n{n\,P_n} $$

we recover the mean value $\langle{n}\rangle$. The point here is that, inside a sum at least, it makes sense to multiply a value by the probability of obtaining that value.

In the continuous case, the probability density function $\rho(x)$ does not give us the probability of $x$, a meaningless concept for a continuous variable (any specific value is infinitesimally unlikely), but it can be integrated over some region to get the probability of the value appearing in that region.

The integral over all values of $x$:

$$
\langle x \rangle =
\int_{-\infty}^{+\infty}
x\,\rho(x)
\,dx
$$

is the continuous equivalent of $(1)$ and gives the mean value of a large set of measurements of $x$. If we think of all the values of $x$ as a cloud of matter that is more or less densely concentrated here or there, $\langle x \rangle$ is like its centre of mass.

But $\rho(x)$ may be symmetrical around the origin and vanish at the origin, e.g. two peaks on either side, making $\langle x \rangle = 0$ despite $x$ never taking the value $0$; so if we are required to call it "the expectation value", we must always remember that it may be a value that never occurs.
